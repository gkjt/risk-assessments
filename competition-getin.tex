\documentclass[12pt,a4paper]{scrartcl}
\usepackage{url, pdflscape}
\usepackage[final]{pdfpages}
\usepackage[colorlinks]{hyperref}
\usepackage{longtable}
\usepackage[margin=0.5in]{geometry}
\title{Student Robotics Risk Assessment Form}

\begin{document}
\maketitle

\begin{description}
\item[Activity being assessed:] Get In for Student Robotics Competition (XXXX, 2020)
\item[Location:] Southampton University Students Union (SUSU)
\item[Persons at risk:] Blueshirts, Technical Crew, SUSU Staff
\end{description}

\begin{description}
\item[Assessor's name:] Thomas Scarsbrook
\item[Responsible Persons:] Health and Safety Lead; Team Leaders
\item[Date of assessment:] \date{\today}
\end{description}
\clearpage

\newcommand{\risk}[5]{
 & #1 & #2 & #3 & #4 & #5 \\
 \hline
}

\begin{landscape}
\section{Risks}
The following risks have been considered for the Student Robotics Competition. 
Further description of the meaning of risk ratings (presented in this section as
$L \times S$) can be found in the next section.

%\centering
\begin{longtable}{|p{8em}|p{12em}|p{4em}|p{15em}|p{8em}|p{4em}|}
\hline
\textbf{Hazard} & \textbf{Harm} & \textbf{Risk Groups} & \textbf{Control Measures} & \textbf{Responsible Person} & \textbf{Risk Rating} \\
\hline
\endhead

\endfoot
Carrying Heavy Loads
\risk{Short and long term back injury from incorrect lifting technique}
{B, TC}
{
\begin{itemize}
	\item All Blueshirts should have received H\&S training which includes manual handling.
	\item Crew must ensure that they follow proper lifting procedure, reminding others when necessary.
	\item Ensure that the lift is planned beforehand, including route and destination and communicate this to all involved.
\end{itemize}
}
{Health and Safety Lead}
{16}



\end{longtable}
\end{landscape}

\input{assessment-guidance}

%\clearpage

%\newpage
%\includepdf[scale=1.0,landscape]{extras/Fire1.pdf}

\end{document}

